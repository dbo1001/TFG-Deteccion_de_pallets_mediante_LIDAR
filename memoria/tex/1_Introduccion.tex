\capitulo{1}{Introducción}

\section{Introducción}
Cada vez es más la tecnología que se implementa en la industria de cara a mejorar sus procesos haciendolos más eficientes. \\Dentro del transporte de mercancias, ya sea a nivel interno como a nivel externo de una industria, se han visto importantes mejoras en tecnología en los ultimos años, como la popularización de los AGV's(Automated Guided Vehicle). \\ \\
Los AGV's son vehiculos automatizados que se usan para transportar cargas, por lo general paletizadas,  a traves de las zonas de las fábricas o almacenes.
Normalmente, estos vehículos tienen rutas preprogramadas para recoger la mercancía, pero ¿qué pasa si la mercanciía que iban a recoger no se encuentra exactamente donde tenía que estar?\\s \\
Una desviación en el ángulo de un palet, una separación de unos metros respecto de donde debía estar, pueden hacer que el AGV no sea capaz de recoger ese pallet, con la consecuencia del paro en esa línea de transporte y la necesidad de una intervención humana para volver a retomar el funcionamiento.\\
Aquí es donde entra en juego la visión artificial, dotar a los AGV de la capacidad de detectar la mercancía y su posición, para así volverlos más precisos y eficientes y no depender de la correcta colocación de la mercancía. \\ \\
Existen multitud de opcionas para el reconocimiento de los pallets en cuanto a sensores se refiere, pero se ha optado por utilizar un sensor de tipo LIDAR por su capacidad de funcionamiento independientemente de las condiciones de luminosidad, funcionando incluso en total oscuridad.\\ Otros sensores como cámaras RGB no son capaces de proporcionarnos este rango de operacion, teniendo el inconveniende de perder mucha precisión en el momento que las condiciones de luminosidad no son las adecuadas y llegando a ser prácticamente inservibles en total oscuridad.\\
En este proyecto se investiga como utilizar un sensor láser LIDAR para el reconocimiento de palets en el entorno simulado de un almacen.\\

\section{Estructura de la memoria}
La memoria se divide según la siguiente estructura:
\begin{itemize}
\tightlist
\item
  \textbf{Introducción:} Descripción  resumida del problema a resolver y solución propuesta al mismo. 
Estructura de la memoria.
\item
  \textbf{Objetivos del proyecto:} Objetivos que pretende alcanzar el proyecto
\item
  \textbf{Conceptos teóricos:} Explicación de los conceptos teóricos no triviales tratados en el proyecto para la comprensión de la solución.
\item
  \textbf{Técnicas y herramientas:} Listado de técnicas metodológicas y
  herramientas utilizadas para gestión y desarrollo del proyecto.
\item
  \textbf{Aspectos relevantes del desarrollo:} Explicación de los aspectos
  destacables que se han dado durante la realización del proyecto.
\item
  \textbf{Trabajos relacionados:} Estado del arte en la investigación de los AGV's autónomos y reconocimiento de materiales paletizados.
\item
  \textbf{Conclusiones y líneas de trabajo futuras:} Conclusiones
  a las que se ha llegado tras la realización del proyecto y potenciales mejoras y lineas de desarrollo futuras.
\end{itemize}
\newpage
Junto a la memoria se proporcionan los siguientes anexos:

\begin{itemize}
\tightlist
\item
  \textbf{Plan del proyecto software:} Planificación temporal y estudio
  de viabilidad del proyecto.
\item
  \textbf{Especificación de requisitos del software:} Se describe la
  fase de análisis; los objetivos generales, el catálogo de requisitos
  del sistema y la especificación de requisitos funcionales y no
  funcionales.
\item
  \textbf{Especificación de diseño:} Se detalla la fase de diseño; el
  ámbito del software, el diseño de datos, el diseño procedimental y el
  diseño arquitectónico.
\item
  \textbf{Manual del programador:} Recoge los aspectos más relevantes
  relacionados con el código fuente (estructura, compilación,
  instalación, ejecución, pruebas, etc.).
\item
  \textbf{Manual de usuario:} Guía de usuario para el correcto manejo de
  la aplicación.
\end{itemize}



\capitulo{2}{Objetivos del proyecto}

Este apartado explica cuales van a ser los objetivos que persigue el proyecto. Tratando de diferenciar los objetivos generales de partida y los objetivos técnicos que aparecen a lo largo del desarrollo.

\section{Objetivos generales}
Los objetivos generales del proyecto se resumen en:
\begin{itemize}
	\item Establecer la conexion con el dispositivo láser para ser capaz de recibir tramas de información desde un ordenador.
	\item Interpretar los datos de lectura recibidos y efectuar el adecuado tratamiento de los mismos.
	\item Concluir si en los datos de lectura existe o no un palet y su posición en el caso de existir.
\end{itemize}

\section{Objetivos técnicos}
Los objetivos técnicos del proyecto son los siguientes:
\begin{itemize}
	\item Desarrollar un algoritmo capaz de tratar los datos recibidos en forma de puntos y procesarlos en tiempo real.
	\item Desarrollar una aplicación en python para la recepción y tratamiento de los datos.
	\item Establecer un umbral para poder decidir si en los datos recibidos se visualiza un palet y conocer la situación del laser respecto del palet en téminos de distancia.
	\item Utilizar la arquitectura Modelo-Vista-Controlador MVC (Model-View-Controller)
	\item Usar durante el desarrollo del proyecto una herramienta de control de versiones, en este caso GitHub
	\item Hacer uso de la metodología ágil Scrum para el desarrollo del proyecto.
\end{itemize}

\section{Objetivos personales}
\begin{itemize}
\item Aprender sobre las distintas aproximaciones al problema de la detección de palets.
\item Hacer uso de los conocimientos aprendidos a lo largo de la carrera.
\item Ser capaz de utilizar herramientas y conocimientos ajenos a la carrera mediante el autoaprendizaje.
\end{itemize}



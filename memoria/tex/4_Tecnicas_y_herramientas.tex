\capitulo{4}{Técnicas y herramientas}

Esta parte de la memoria tiene como objetivo presentar las técnicas metodológicas y las herramientas de desarrollo que se han utilizado para llevar a cabo el proyecto. 


\section{Herramientas para el desarrollo de la memoria}
Para el desarrollo de la memoria del proyecto se ha usado LaTex como procesador de textos, principalmente por su libertad y simplicidad a la hora de manipular los distintos parametros del documento. Además, desde LaTex se puede exportar directamente el archivo a PDF de una manera muy sencilla.
Como desventaja de este programa, cabe destacar el aprendizaje necesario al no haberlo utilizado antes a lo largo del grado en contraposición a la facilidad de haber utilizado un procesador de textos en el que tuviera más experiencia como por ejemplo Microsoft Word u Open Office.

\section{Herramientas para el desarrollo y pruebas del código}

\subsection{Realterm}
Este programa se ha utilizado para la conexión con el dispositivo láser. Su funcionamiento se resume en un programa para tramsmitir mensajes TCP a través del puerto serie. Así, a través de esta conexión, establecemos una via de comunicación con el láser de tipo Cliente-Servidor.


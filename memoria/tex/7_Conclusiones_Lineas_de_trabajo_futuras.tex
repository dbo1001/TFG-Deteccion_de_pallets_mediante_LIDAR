\capitulo{7}{Conclusiones y Líneas de trabajo futuras}

En este apartado se expresan las conclusiones del proyecto y las posibles mejoras y lineas de trabajo futuras para continuar con el proyecto.

\section{Conclusiones}

Después de la realización del proyecto se ha llegado a las siguientes conclusiones:

\begin{itemize}
\item El objetivo del proyecto se ha cumplido con ciertas salvedades.\\
	Se ha conseguido realizar un programa con un algoritmo capaz de detectar la existencia de palets utilizando el sensor láser en tiempo real, mostrando el proceso al usuario y calculando satisfactoriamente la distancia y el angulo a las patas del palet detectado. No obstante las condiciones en las que se puede reconocer un palet no han sido suficientemente probadas en diferentes entorno y condiciones.
La robustez de la detección es suficiente pero mejorable.

\item Usar un sistema láser que nunca antes había utilizado ha supuesto un reto de adaptación y aprendizaje para poder llegar a manejar y comprender su funcionamiento y la transmisión de datos.

\item Se ha aprendido a buscar soluciones a los contratiempos y poder continuar con el desarrollo, algo que ha sido muy común a lo largo de todo el proceso tanto en el entorno de programación como con el algoritmo en sí o las herramientas utilizadas.

\item Se han empleado gran parte de los conocimientos aprendidos a lo largo del grado asi como otros conocimientos que se han obtenido durante la realización del proyecto.

\item Se han descubierto nuevas herramientas y familiarizado con su uso las cuales resultaran útiles para desarrollos futuros.

\item Se ha conocido como llevar a cabo un proyecto de investigación sobre un tema prácticamente desconocido como era la visión artificial. Comprendiendo la estructura y pasos del mismo.

\item Se ha aprendido a realizar una búsqueda sobre el estado del arte de un problema o tema, mediante búsquedas bibliográficas y el descubrimiento de bibliotecas online donde encontrar artículos científicos y más.

\item Se han sintetizado muchos conocimientos del grado que se aprenden de manera aislada. Es decir, se ha obtenido una visión global de como todos esos conocimientos son necesarios y colaboran de manera intrínseca a la hora de llevar a cabo un proyecto real, aunque haya sido de carácter académico.

Recapitulando, el desarrollo de este proyecto aunque en ocasiones frustrante y muy demandante ha sido por lo general un proceso de aprendizaje provechoso en el que se ha descubierto más en detalle los problemas que van surgiendo y como llevar a cabo su investigación y desarrollo.

\end{itemize}

\section{Líneas de trabajo futuras}

\subsection{Mejoras en la detección de palets}
La detección de palets puede ser mejorada para ser más robusta y precisa, realizando un tratamiento de los datos más exhaustivo.\\
Además, se planteó durante el desarrollo la incorporación de una red neuronal en contraposición al algoritmo empleado, haciendo uso de un algoritmo de aprendizaje automatizado, que con una base de datos de palets fuera entrenado para después funcionar como clasificador.

\subsection{Desarrollo de una interfaz gráfica}
Aunque el destino de la aplicación vaya a ser su instalación en un AGV, queda abierto a desarrollos futuros la incorporación de una interfaz gráfica que permita al operario comprender y utilizar de una manera más sencilla e intuitiva el programa. 

\subsection{Seguridad del código}
En futuras iteraciones se puede mejorar la robustez del código frente a excepciones, valores fuera de rango etc.

\subsection{Prueba en un entorno real}
Por limitaciones en el tiempo y lugar donde se ha realizado el proyecto, no se ha podido probar el funcionamiento del laser y el programa en un entorno real de una fábrica y almacen, sino que se ha simulado el encuentro del laser con un palet en un laboratorio, lejos de presentar las condiciones a las que se enfrentará en un entorno real.






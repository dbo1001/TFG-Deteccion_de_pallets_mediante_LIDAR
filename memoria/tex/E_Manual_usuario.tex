\apendice{Documentación de usuario}

\section{Introducción}
En este manual se explica cómo instalar la aplicación y cómo utilizarla correctamente.

\section{Requisitos de usuarios}

Para poder utilizar la aplicación, se debe tener un hardware específico si lo que se quiere es realizar lecturas con el equipo láser. Existe la posibilidad de trabajar en modo local sin necesidad de la conexión con el láser, utilizando el servidor de prueba que se suministra con varias tramas, no obstante, no es el objetivo final de la aplicación sino un paso intermedio para su correcto desarrollo.\\
Se necesita el siguiente material:\\
\begin{itemize}
\item Un ordenador en el que poder ejecutar el programa. Es necesario que disponga de un puerto ethernet. Debe cumplir ciertos requerimientos software que se detallan en el apartado instalación \ref{section:instalacion}. Para la ejecución del IDE Spyder contenido en el entorno Anaconda se recomienda cómo \href{https://docs.anaconda.com/anaconda/install/}{mínimo}:
\begin{itemize}
\tightlist


\item Sistema operativo: Windows 7 or superior, 64-bit macOS 10.10+, o Linux, incluyendo Ubuntu, RedHat, CentOS 6+, y otros.
\item Disco duro: Mínimo 5GB de espacio libre para la descarga e instalación.
\item Memoria: 4GB de memoria RAM
\end{itemize}


\item El equipo láser para realizar las lecturas de puntos. El modelo utilizado para el desarrollo (Hokuyo Safety Laser Scanner /UAM-05LP-T301).
\item Fuente de alimentación DC de 24v necesaria para suministrar al láser de energía eléctrica continua.
\item Cable de red Ethernet para efectuar el intercambio de mensajes con el equipo láser.
\end{itemize}
\section{Instalación}\label{instalacion}
Para poder ejecutar el programa es necesario haber instalado antes los siguientes elementos software:\\
\begin{itemize}
\tightlist
\item Python
\item Bibliotecas necesarias (Sklearn, matplotlib, numpy)
\item Opcional: Un entorno de desarrollo integrado. Se recomienda Spyder.
\end{itemize}

Las instrucciones de instalación de estos componentes se encuentran detalladas en el Anexo D - Manual del programador \ref{section:instrucciones}

\section{Manual del usuario}

Para comenzar la ejecución de la aplicación, una vez completados los pasos de instalación, tenemos dos opciones disponibles: la ejecución desde línea de comandos o la ejecución desde un IDE compatible con Python. \\
No es necesaria la modificacíon de ningún parámetro por parte del usuario, en sistema viene confirgurado automáticamente. \\
El sistema genera una gráfica que se va actualizando conforme recibe las tramas del láser en tiempo real y devolviendo la decisión sobre la existencia o no de palet en la trama actual en pantalla.

\subsection{Ejecución desde la línea de comandos}
Primero debemos abrir una terminal (cmd) y navegar hasta la ruta que contiene los ficheros Python (source).\\
Una vez estemos en ese directorio y asegurándose de que el equipo láser esté encendido y conectado, introducir la siguiente línea en el cmd y ejecutarla: \textit{objectDetectorViaLaser.py}.\\
En el caso de querer trabajar con el servidor local, es necesario previo al anterior paso ejecutar la siguiente línea en otra terminal a parte: \textit{DummySV.py}.
\imagen{paso1consola}{Navegación al directorio 'source'}
\imagen{paso2consola}{Ejecución del archivo}

\subsection{Ejecución desde un IDE}
Existen multitud de aplicaciones compatibles con Python. En este proyecto se ha trabajado con Spyder.\\
Una vez instalada la aplicación deseada siempre y cuando sea compatible, abrimos en el IDE los archivos necesarios (\textit{objectDetectorViaLaser.py y DummySV.py en el caso de querer trabajar en local}) y lanzamos el primero de ellos.\\


\imagen{paso1ide}{Apertura de los archivos}
\imagen{paso2ide}{Ejecución de la aplicación}